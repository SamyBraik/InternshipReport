% Beamer presentation for "Synthetic turbulent velocity field using statistics"
% Author: Samy Braik (adapted to Beamer)

\documentclass[11pt]{beamer}
\usetheme{Madrid}
\usecolortheme{dolphin}
\usepackage[utf8]{inputenc}
\usepackage[T1]{fontenc}
\usepackage{amsmath,amssymb,amsthm}
\usepackage{graphicx}
\usepackage{booktabs}
\usepackage{siunitx}
\usepackage[backend=biber,style=numeric,citestyle=numeric]{biblatex}
\addbibresource{references.bib}
\usepackage{comment}
%\usepackage[hidelinks]{hyperref}

% Title
\title[Synthetic turbulent field]{Synthetic turbulent velocity field using statistical method}
\author[Samy Braik]{Samy Braik \\[0.5em] \small Supervisor: Aurélien Larcher\inst{1}, Jonathan Viquerat\inst{1}, Fabien Duval\inst{2} and Aubin Brunel\inst{2}}
\institute{
  \inst{1} CEMEF, Mines Paris – PSL \\
  \inst{2} ANSR
}\date{\today}

\begin{document}

\begin{frame}
  \titlepage
\end{frame}

\begin{frame}{Outline}
  \tableofcontents
\end{frame}

\section{Introduction}
\begin{frame}{What is turbulence?}
  \begin{itemize}
    \item Turbulence: complex, aperiodic fluid motion with strong vortical structures.
    %\item Contrasts with laminar flow; typically characterized by high Reynolds number $\mathrm{Re}$.
    \item We study \emph{Homogeneous and Isotropic Turbulence (HIT)}: statistics invariant under space translations and rotations.
  \end{itemize}
  \vfill
  \begin{figure}
    \centering
    \includegraphics[width=0.5\textwidth]{illustrations/TurbulenceExample.jpg}
    \caption{Slice through scalar dissipation (CNRS UMR 6614 CORIA and JSC).}
  \end{figure}
\end{frame}

\section{Model}

\subsection{Random Fourier model}
\begin{frame}{Random Fourier model}
  \begin{itemize}
    \item Build a synthetic velocity field by summing random Fourier modes.
    \item Frozen-turbulence assumption (no time dependence):
  \end{itemize}
  \begin{equation}
    u^s(x)= 2\sum_{n=1}^{N} \hat{u}_n \cos(\kappa^n\cdot x + \psi_n)\,\sigma^n
  \end{equation}
  \begin{description}
    \item[$\kappa^n$] wave vector (random on a half-sphere to preserve isotropy)
    \item[$\sigma^n$] direction (divergence-free: $\kappa^n\cdot\sigma^n=0$)
    \item[$\psi_n$] random phase (uniform $\mathcal{U}[0,2\pi]$)
    \item[$\hat{u}_n$] amplitude linked to prescribed energy spectrum $E(\kappa_n)$
  \end{description}
\end{frame}

\begin{frame}{Coefficient sampling}
  \begin{itemize}
    \item Wave vector components (spherical coordinates):
        \begin{align}
            \kappa_1 &= \sin(\theta)\cos(\varphi) \\
            \kappa_2 &= \sin(\theta)\sin(\varphi) \\
            \kappa_2 & = \cos(\theta)
        \end{align}
    \item Sampling densities: $f_\theta(\theta)=\frac{\sin\theta}{2}$, $f_\varphi(\varphi)=\frac{1}{2\pi}$.
    \item Direction vector $\sigma$ obtained by fixing an angle $\alpha\sim\mathcal{U}[0,2\pi]$ and ensuring $\kappa\cdot\sigma=0$.
        \begin{align}
            \sigma_1&=\cos(\varphi)\cos(\theta)\cos(\alpha)-\sin(\varphi)\sin(\alpha) \\
            \sigma_2&=\sin(\varphi)\cos(\theta)\cos(\alpha)+\cos(\theta)\sin(\alpha) \\
            \sigma_3&=-\sin(\theta)\cos(\alpha)
        \end{align}
  \end{itemize}
\end{frame}

\subsection{Energy spectrum}
\begin{frame}{Energy spectrum: Kolmogorov and VKP}
  \begin{itemize}
    \item Kolmogorov inertial-range law: $E(\kappa)=C_k\,\varepsilon^{2/3}\,\kappa^{-5/3}$ (valid in the inertial range).
    \item To capture large- and small-scale behavior use a full model such as von K\'arm\'an–Pao (VKP):
      \begin{equation}
        E_{\mathrm{VKP}}(\kappa)=\frac{2}{3}\alpha_e\,\kappa L_e\frac{(\kappa L_e)^4}{[(\kappa L_e)^2+1]^{17/6}}\exp\big(-2(\kappa L_\eta)^2\big)
      \end{equation}
    \item Mode amplitudes: $\hat{u}_n=\sqrt{E(\kappa_n)\,\Delta\kappa_n}$ (log-spacing used for $\kappa_n$).
  \end{itemize}
  \begin{figure}
    \centering
    \includegraphics[width=0.6\textwidth]{illustrations/Energy_Spectrum_VKP.png}
    \caption{VKP spectrum vs theoretical reference}
  \end{figure}
\end{frame}

\section{Quality metrics \& Limitations}
\begin{frame}{Metrics to assess synthetic field}
  \begin{itemize}
    \item Energy spectrum reconstruction: match between theoretical and reconstructed $E(\kappa)$.
    \item Component means close to zero (homogeneity).
    \item RMS speed must match prescribed value.
    \item Velocity increments statistics (non-Gaussian heavy tails expected at small scales).
  \end{itemize}
\end{frame}

\begin{frame}{Empirical base metrics}
  \begin{table}
    \centering
    \begin{tabular}{lr}
      \toprule
      \textbf{Direction} & \textbf{Mean} \\
      \midrule
      x & -0.00269 \\
      y & -0.00010 \\
      z & -0.00126 \\
      \bottomrule
    \end{tabular}
    \qquad
    \begin{tabular}{lr}
      \toprule
      \textbf{Direction} & \textbf{RMS (expected: 0.222)} \\
      \midrule
      x & 0.19583 \\
      y & 0.17599 \\
      z & 0.19307 \\
      \bottomrule
    \end{tabular}
    \caption{Velocity mean and RMS}
  \end{table}
\end{frame}

\begin{frame}{Velocity increments \& heavy tails}
  \begin{itemize}
    \item Velocity increments: $\delta_r u = u(x+r)-u(x)$.
    \item In random Fourier base model increments are typically Gaussian (kurtosis $=3$).
    \item Real turbulence: increments show heavy tails (increasing kurtosis when $r\to 0$).
  \end{itemize}
  \begin{figure}
    \centering
    \includegraphics[width=0.45\textwidth]{illustrations/TransVelIncrExample.png}
    \includegraphics[width=0.45\textwidth]{illustrations/LongVelIncrExample.png}
    \caption{Transverse and longitudinal increments (model vs Gaussian)}
  \end{figure}
\end{frame}

\begin{frame}{Random Fourier Velocity field}
  \begin{figure}
    \centering
    \includegraphics[width=0.6\textwidth]{illustrations/Velocity_Example.png}
    \caption{Synthetic velocity field obtain via random Fourier model}
  \end{figure}
\end{frame}

\section{Method}
\begin{frame}{Goal and strategy}
  \begin{itemize}
    \item \textbf{Goal:} produce heavy-tailed velocity increments while preserving energy spectrum and isotropy.
    \item \textbf{Strategy:} parameterize coefficients (angles, phases, amplitudes) as learnable parameters and optimize a loss.
    %\item Use PyTorch \texttt{nn.Parameter} for learning coefficients directly.
  \end{itemize}
\end{frame}

\begin{frame}{Losses}
  \begin{block}{Flatness (kurtosis) objective}
    \begin{itemize}
      \item Use kurtosis (or "flatness" = kurtosis $-$ 3) of velocity increments as a loss term to encourage heavy tails.
    \end{itemize}
  \end{block}
  \begin{block}{Energy-spectrum objective}
    \begin{itemize}
      \item MSE between reconstructed and target spectrum, optionally with a smoothness regulariser (penalize high-frequency oscillations in $E(\kappa)$).
    \end{itemize}
  \end{block}
  \vfill
  \centering Combined loss: $\mathcal{L}=\lambda_F\mathcal{L}_F + \lambda_{ES}\mathcal{L}_{ES} + \text{regularisers}$
\end{frame}

\subsection{Preliminary experiments}
\begin{frame}{Preliminary experiments}
  \begin{itemize}
    \item Tested optimization toward Gaussian target flatness (flatness $=0$) from different initialization.
    \item Starting from Gaussian increments: parameters remain close to initial Janin et al. choice.
    \item Starting from heavy-tailed increments: $\theta$ adapt to a flatter distribution.
  \end{itemize}
  \begin{figure}
    \centering
    \includegraphics[width=0.9\textwidth]{illustrations/StartGaussian.png}
    \caption{Angle distributions starting from Gaussian increments}
  \end{figure}
\end{frame}

\begin{frame}{Preliminary experiments}
  \begin{figure}
    \centering
    \includegraphics[width=0.9\textwidth]{illustrations/StartHeavyTail.png}
    \caption{Angle distributions starting from heavy-tailed increments}
  \end{figure}
  \begin{itemize}
    \item Observations: RMS and means largely preserved while increment statistics change.
  \end{itemize}
\end{frame}

\section{Results \& Discussion}

\begin{frame}

\end{frame}

\begin{frame}{Distribution shift}
    \begin{figure}
        \centering
        \includegraphics[width=1\linewidth]{illustrations/AnglesDistributionLearned.png}
        \caption{Learned angles distributions}
    \end{figure}
\end{frame}

\begin{frame}{Key observations}
  \begin{itemize}
    \item The Random Fourier model can be steered (via angles) to modify increment statistics.
    \item Trade-offs observed: stronger heavy tails sometimes perturb RMS.
    %\item Phase distribution and angle spread are influential on small-scale statistics.
  \end{itemize}
\end{frame}

\begin{comment}
\section{Conclusion \& Future Work}
\begin{frame}{Conclusion}
  \begin{itemize}
    \item Random Fourier is a flexible and interpretable model for synthetic HIT.
    \item Optimising coefficients can introduce realistic intermittency (heavy tails) while preserving primary statistics.
    \item Regularisation and composite losses are crucial to maintain spectral fidelity and isotropy.
  \end{itemize}
\end{frame}

\begin{frame}{Future work}
  \begin{itemize}
    %\item Explore richer parameterization (e.g. correlated phases, amplitude stochasticity).
    \item Incorporate time dependence (non-frozen turbulence) and match temporal spectra.
    \item Compare generated fields against DNS at multiple Reynolds numbers and compute higher-order structure functions.
    \item Potential use of generative models conditioned on Reynolds number or integral scales.
  \end{itemize}
\end{frame}
\end{comment}

\appendix
\section{Appendix: Turbulence parameters}
\begin{frame}{Parameters used}
  \begin{center}
  \begin{tabular}{lll}
  \toprule
  \textbf{Parameter} & \textbf{Value} & \textbf{Unit}\\
  \midrule
  Length of box & $\tfrac{\pi}{6}$ & \si{\meter}\\
  Number of modes & 250 or 500 & --\\
  RMS speed & 0.222 & \si{\meter\per\second}\\
  Integral length scale & 0.024 & \si{\meter}\\
  Viscosity & $1.8\times10^{-5}$ & \si{\meter^2\per\second}\\
  Minimum wave number & $\tfrac{2\pi}{1.0}$ & \si{\per\meter}\\
  Maximum wave number & $\tfrac{2\pi}{0.01}$ & \si{\per\meter}\\
  \bottomrule
  \end{tabular}
  \end{center}
\end{frame}

\begin{frame}[allowframebreaks]{References}
\printbibliography
\end{frame}

\end{document}
