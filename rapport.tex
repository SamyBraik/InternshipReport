\documentclass[a4paper,12pt]{article}

% Packages
\usepackage[utf8]{inputenc}
\usepackage[T1]{fontenc}
\usepackage{geometry}
\usepackage{graphicx}
\usepackage[hidelinks]{hyperref}
\usepackage{amsmath}
\usepackage{amssymb}
\usepackage{bbm}

\usepackage[backend=biber,style=numeric,citestyle=numeric]{biblatex}
\addbibresource{references.bib}

% Page setup
\geometry{margin=1in}

% Title
\title{Synthetic turbulent velocity field using statistics}
\author{Samy Braik}
\date{\today}

\begin{document}

\maketitle

%\begin{abstract}
% Brief summary of your report.
%\end{abstract}

\newpage
\tableofcontents
\newpage

\section{Introduction}
Turbulence is the term used to describe the unpredictable and complex behavior of a fluid's motion. It is characterized by vortex. It is opposed to a laminar flow, where the particles' movement is pre. Formally, the behavior of a fluid is quantified by the Reynolds number (Re), the higher this number is the more turbulence the fluid is. 

\bigskip

In this work, we study an ideal type of turbulence flow called Homogeneous and Isotropic Turbulence (HIT). \\
There are few properties that characterized this type of flow. It is homogeneous which means that the statistical properties (mean, variance, correlation functions) do not depend on position in space, and it is isotropic which means that these statistics are invariant under rotations or reflections of the coordinate.\\

%First, the spatial average should be 0 for each component. Since the flow is isotropic, the turbulence properties should remain invariant to translation or axis rotations.

\bigskip

When the goal is to use statistical methods to produce turbulent velocity field two paths are worth considering. The first one is to simply use a statistical method that generate the velocity field directly. This is done in \cite{Yousif_Yu_Lim_2022},\cite{wang2025fourierflowfrequencyawareflowmatching} or \cite{parikh2025conditionalflowmatchinggenerative}. \\
Here, generative models seem the most suitable. Two major difficulties could be encountered, the huge cost and the poor generalization of the method. Indeed, in order to train such a model, there is a need for highly resolved turbulence usually produced with Direct Numerical Simulation (DNS) which are costly numerical methods. Apart from that, there is the natural cost of the method itself that can't be sidelined. The poor generalization of the method stem from the fact that usually, the models are trained on fields with a specific Reynolds number which leads to poor robustness if the turbulence parameters are changed. \\
The other path would be to start from an already existent method. For example in the case of Reynolds-averaged Navier–Stokes equations (RANS) technique (see \cite{} for details), we can use statistical methods to learn a closure model, like it is done in \cite{Bezgin2021}. This allows a better theoretical guarantee on the produced and also get rid of the robustness problem from the previous case.  

\newpage

\section{Model}

\subsection{Random Fourier}

We set ourselves in the random Fourier model developed in \cite{Janin2021} and briefly remind it. 
Some papers that study turbulence start from a velocity field generated by DNS and use Fourier transform to study the field properties in the spectral space. The motivation behind this model, is to perform an inverse Fourier transform to generate a synthetic turbulent field with a good choice of Fourier coefficient. \\
This leads to the following expression of the velocity field 

\begin{align}
    u^s(x,t)=\int_{-\infty}^\infty\int_{-\infty}^{\infty}\left[ \hat{u}(\kappa,\omega)e^{l\psi(\kappa,\omega)}\sigma(\kappa,\omega) \right] e^{l(\kappa\cdot x + \omega t)} d\kappa d\omega
\end{align}
Discretizing the expression in N random Fourier modes and considering that the field is real it could be written 
\begin{align}
    u^s(x,t)= 2 \sum_{i=1}^N \hat{u}_n \cos(\kappa^n\cdot x + \psi_n + \omega_n t)\sigma^n
\end{align}
Furthermore, we place ourselves in the following work in a froze turbulence which means that we ignore the effect of time. Therefore, the expression if simplified and we have
\begin{align}
    u^s(x,t)= 2 \sum_{n=1}^N \hat{u}_n \cos(\kappa^n\cdot x + \psi_n)\sigma^n
\end{align}
For each $n^{th}$ Fourier mode associated with the wave vector $\kappa^n$, $\hat{u}_n$ is the amplitude, $\psi_n$ the phase, $\sigma^n$ the direction and $\omega_n$ the time-frequency. \\
Like previously mentioned, a good choice of the Fourier coefficient is needed in order to produce a realistic velocity field. \cite{Janin2021} features a discussion on the choice of $\hat{u}, \kappa, \psi_n$ and $\sigma$.

\subsection{Coefficient choice}
To ensure the statistical isotropy of the generated field, the wave vector $\kappa$ is chosen randomly on the half-sphere.
This leads to the following components choices
\begin{align}
    \kappa_1 &= \sin(\theta)\cos(\varphi) \\
    \kappa_2 &= \sin(\theta)\sin(\varphi) \\
    \kappa_2 & = \cos(\theta)
\end{align}
with the angles chosen randomly according to the following probability density functions $f_\theta(x)=\frac{sin(x)}{2}$ and $f_\varphi(x)=\frac{1}{2\pi}\mathbbm{1}_{[0,2\pi]}$. \\
Regarding the choice of $\sigma$ it is such that the divergence-free condition $\nabla\cdot u^s=0$ which implies that $\kappa\cdot\sigma=0$. \\
Therefore, 
\begin{align}
    \sigma_1&=\cos(\varphi)\cos(\theta)\cos(\alpha)-\sin(\varphi)\sin(\alpha) \\
    \sigma_2&=\sin(\varphi)\cos(\theta)\cos(\alpha)+\cos(\theta)\sin(\alpha) \\
    \sigma_3&=-\sin(\theta)\cos(\alpha)
\end{align}
with $\alpha$ chose according to a uniform distribution on $[0,2\pi]$ i.e. $f_\alpha(x)=\frac{1}{2\pi}\mathbbm{1}_{[0,2\pi]}$.

\subsection{Energy spectrum}
In a turbulent setup, an energy cascade is observed where a transfer of energy from large eddies to small scales (or sometimes the opposite). To capture this energy transfer, 


\begin{align}
    E = C_k \kappa^{-5/3}\varepsilon^{2/3}
\end{align}
where $\varepsilon$ is the turbulent dissipation, $\kappa$ the wave number and $C_k$ the Kolmogorov constant.

\bigskip
We use the von-Kármán Pao (VKP) energy spectrum to try to reconstruct the theoretical energy spectrum. \\
It is defined by 

\begin{align}
    E_{\text{VKP}}(\kappa)=\frac{2}{3}\alpha_e k L_e \frac{(kL_e)^4}{[(kL_e)^2+1]^{17/6}}\exp(-2(\kappa L_\eta)^2)
\end{align}

\subsection{Limitations and goals}
The assessment of the synthetic velocity field's quality could be done by looking at few metrics. First, the reconstructed energy spectrum should match as close as possible the theoretical spectrum. Then, for each component, the average should be 0 and the Root Mean Square (RMS) values should match the one used to build the turbulence in the first place. Lastly, we look at the shape of the velocity increments.   

\bigskip
Although, the model retrieve the correct energy spectrum and limited anisotropy, the velocity increments obtained are Gaussian. It is known that velocity increments appear to be non-Gaussian and more particularly heavy-tailed distribution. For instance \cite{} [Trouver un papier qui montre que pour un certain niveau de turbulence (Reynolds disons), regarder un r particulier et montre un beau plot].

\bigskip
With the random Fourier model in mind, the goal is to identify which coefficient could lead to more realistic velocity increments without disrupting the other good properties that are already satisfied. 


\section{Method}
To tackle the problem, no obvious path stood out. The work was highly exploratory and featured trials and errors along the way.

\subsection{Phase coefficient}
The first idea was to work on the phase parameter $\psi$. 


\subsection{Phase, wave vector and direction}

\subsection{Angles}


\section{Results and Discussion}
% Results and analysis.

\section{Conclusion}
% Conclusion and future work.

\printbibliography

\end{document}